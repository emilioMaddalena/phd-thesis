%\begingroup
%\let\cleardoublepage\clearpage


% English abstract
\cleardoublepage
\chapter*{Abstract}
\markboth{Abstract}{Abstract}
\addcontentsline{toc}{chapter}{Abstract (English/Français)} % adds an entry to the table of contents
% put your text here
This thesis is situated at the crossroads between machine learning and control engineering. Our contributions are both theoretical, through proposing a new uncertainty quantification methodology in a kernelized context; and experimental, through investigating the suitability of two machine learning techniques to integrate feedback loops in two challenging real-world control problems.

The first part of this document is dedicated to deterministic kernel methods. First, the formalism is presented along with some widespread techniques to craft surrogates for an unknown ground-truth based on samples. Next, standing assumptions are made on the ground-truth complexity and on the data noise, allowing for a novel robust uncertainty quantification (UQ) theory to be developed. By means of this UQ framework, hard out-of-sample bounds on the ground-truth values are computed through solving convex optimization problems. Closed-form outer approximations are also presented as a lightweight alternative to solving the mathematical programs. Several examples are given to illustrate how the control community could benefit from using this tool.

In the second part of the thesis, statistical models in the form of Gaussian processes (GPs) are considered. These are used to carry out a building temperature control task of a hospital surgery center during regular use. The engineering aspects of the problem are detailed, followed by data acquisition, the model training procedure, and the developed predictive control formulation. Experimental results over a four-day uninterrupted period are presented and discussed, showing a gain in economical performance while ensuring proper temperature regulation. 

Lastly, a specialized neural network architecture is proposed to learn linear model predictive controllers (MPC) from state-input pairs. The network features a parametric quadratic program (pQP) as an implicit non-linearity and is used to reduce the storage footprint and online computational load of MPC. Two examples in the domain of power electronics are given to showcase the effectiveness of the proposed scheme. The second of them consists in enhancing the start-up response of a real step-down converter, deploying the learned control law on an 80$\,$MHz microcontroller and performing the computations in under $30\,$ microseconds.

\smallbreak
\textbf{Keywords:} kernel learning, Gaussian processes, model predictive control, neural networks, power electronics, building control.

%
%% German abstract
%\begin{otherlanguage}{german}
%\cleardoublepage
%\chapter*{Zusammenfassung}
%\markboth{Zusammenfassung}{Zusammenfassung}
%% put your text here
%\lipsum[1-2]
%\end{otherlanguage}

% French abstract
\begin{otherlanguage}{french}
\cleardoublepage

\chapter*{Résumé}
\markboth{Résumé}{Résumé}

Cette thèse se situe à la croisée de l'apprentissage automatique et de l'ingénierie des systèmes de contrôle. Nos contributions sont théoriques, en proposant une nouvelle méthodologie de quantification des incertitudes, ainsi que expérimentales, en étudiant l'adéquation de certaines techniques d'apprentissage automatique à intégrer des boucles de rétroaction dans le contexte de deux problèmes de contrôle réels et difficiles.

La première partie du document est consacrée aux méthodes noyau déterministes. Tout d'abord, le formalisme est présenté ainsi que certaines techniques pour créer des modèles pour une fonction inconnue à partir d'échantillons. Ensuite, des hypothèses sont faites sur la complexité de la fonction inconnue et sur le bruit affectant les données, ce qui permet de développer une nouvelle théorie robuste de quantification d'incertitude. Grâce à cette procédure de quantification de l'incertitude, des limites supérieures et inférieures hors échantillon sur les valeurs de la fonction peuvent être calculées en résolvant des problèmes d'optimisation convexes. Des approximations extérieures sous forme fermée sont présentées comme une alternative à la résolution des problèmes d'optimisation. Plusieurs exemples sont donnés pour illustrer comment la communauté du contrôle pourrait se bénéficier de l'utilisation de cet outil.

Dans la deuxième partie de la thèse, nous utilisons des modèles statistiques sous forme de processus Gaussiens sont considérés. Ceux-ci sont employés pour réaliser une tâche de contrôle de la température d'un centre de chirurgie hospitalier étant régulièrement utiliser. Les aspects techniques du problème sont détaillés, suivis de l'acquisition des données, de la procédure d'apprentissage du modèle et de la formulation de la commande prédictive développée. Les résultats expérimentaux sur une période ininterrompue de quatre jours sont présentés et discutés, montrant un gain de performance économique tout en assurant une régulation adéquate de la température.

Le dernier chapitre présente une architecture de réseau neuronal spécialisée pour apprendre des contrôleurs prédictifs basés sur des modèles linéaires à partir de paires état-commande. Le réseau dispose d'un programme quadratique paramétrique comme couche non-linéaire implicite et est utilisé pour réduire le besoin de stockage et la charge computationnelle en-ligne des contrôleurs prédictifs. Deux exemples dans le domaine de l'électronique de puissance sont donnés pour démontrer l'efficacité du schéma proposé. Le deuxième exemple consiste dans l'amélioration de la réponse du démarrage d'un convertisseur abaisseur réel, où la loi de commande a été déployer dans un microcontrôleur de 80 MHz et les calculs ont été effectuer en moins de 30 microsecondes.


\smallbreak
\textbf{Mots-clés :} méthodes à noyaux, processus gaussien, commande prédictive, réseaux neuronaux, électronique de puissance, contrôle des bâtiments.

\end{otherlanguage}


%\endgroup			
%\vfill
